\documentclass[11pt,letter]{moderncv}

\moderncvtheme[blue]{classic} 
%\usepackage[utf8]{inputenc}  %Windows 

%\usepackage[scale=0.975]{geometry}
\usepackage[top=1.3cm, bottom=1.3cm, left=0.8cm, right=0.8cm]{geometry}
\usepackage{graphicx}

\firstname{Stephan X.}
\familyname{Esterhuizen}
\title{Signal Analysis Engineer}         
\address{335 - Tracking Systems and Applications}{}   
\mobile{626-319-4871}
%\phone{}                
%\fax{Votre Fax}                      
\email{stephan.esterhuizen@jpl.nasa.gov}
\extrainfo{}
%\photo[50pt]{photo.png} 
\quote{Objective: To work on interesting problems for missions beyond Earth.}
\makeatletter
\renewcommand*{\bibliographyitemlabel}{\@biblabel{\arabic{enumiv}}}
\makeatother

%\usepackage{multibib}
%\newcites{book,misc}{{Books},{Others}}

\nopagenumbers{}                         
\begin{document}
\maketitle
\section{Education}
\cventry{2004--2006}{M.S. Electrical Engineering}{University of Colorado}{Boulder}{}{}
\cventry{2000--2004}{B.S. Computer Engineering}{University of Colorado}{Boulder}{}{}

\section{Skills}
\cvline{Languages}{C/C++, Python, Matlab, Bash, Verilog, Assembly, JavaScript, HTML/CSS, SQL}
\cvline{Other}{Linux, RTLinux, VxWorks, RTEMS, Django Frameworks, Github, OpenGL, Blender, ZeroMQ}
\cvline{Keywords}{Software/Firmware/Hardware engineering, hacker, reverse engineer, amateur radio license, scripting, microwave radios, radiometric tracking, satellite cross-links, one-way and two-way ranging, GPS, Global Navigation Satellite Systems (GNSS), signal processing, DSN, web scraping, laser stabilization, device drivers, websites, user-interface coding, embedded systems, science instruments, circuit board design, ephemeris, orbits, MAROS/MPX}

\section{Professional Experience (JPL 2006-present)}
\cventry{2015--present}{Principal Investigator}{8x R\&TD}{PI for JPL R\&TD - Towards sub-millimeter formation knowledge and millimeter-level formation control for small satellites. Developing software, hardware, and algorithms to enable fuel-efficient formation flight with small spacecraft. Using Real Time GIPSY (RTGx) processing GNSS and spacecraft crosslinks}{}{}

\cventry{2015--present}{Software Engineer}{ExoMars, InSight}{Working with India's Giant Metrewave Radio Telescope (GMRT) Engineering team to support direct-to-earth (DTE) detection of UHF signals during Entry Descent and Landing (EDL). Designing custom signal processing to run on GMRT's Linux cluster for real-time detection of signals from Mars}{}{}

\cventry{2015--present}{Software Engineer}{BepiColombo}{Advanced Ranging Initiative (ARI) - developing tracking software running at the DSN. Software will use Pseudo Noise (PN) codes transmitted from the DSN and coherently transponded by BepiColombo to measure the range to the spacecraft at the 15cm level}{}{}


\cventry{2015--2016}{Co-Investigator}{SMAP Reflections}{Received Spontaneous R\&D money, developed signal processing to detect Earth-reflected GPS L2 signals using SMAP radar receiver (post SMAP transmitter anomaly)}{}{}

\cventry{2015--2016}{Technical Team Lead}{CICERO}{Led a team of 5 JPL Engineers to deliver a radio occultation instrument for the CICERO cube-sat mission. On-time and on-budget delivery of instrument}{}{}

\cventry{2014--present}{Co-Investigator}{ACT RFASIC}{Designing a new Radio Frequency (RF) ASIC for GNSS remote sensing  applications. Architect of optimal RF ASIC for radiometric performance}{}{}

\cventry{2012--present}{US. Delegate}{}{Part of the US delegation representing NASA at the United Nations Office of Outer Space Affairs, International Committee on GNSS. Advising UN member states on designing inter-operable GNSS constellations with the scientific community in mind}{}{}

\cventry{2012}{Software Engineer}{MSL}{Developed software for detecting (on Earth) the Mars rover Curiosity's 10W signal as transmitted from Mars during EDL using the 64 meter Parkes Radio Telescope}{}{}


\cventry{2011-2012}{Software Engineer}{GRAIL}{Software architect, intercepting the GRAIL inter-spacecraft link using a 34-meter radio telescope to demonstrate sub 100 nanosecond time transfer from the Moon to Earth}{}{}

\cventry{2011,2015}{Field Testing}{White Sands Missile Range (WSMR)}{Part of team field testing various GNSS receivers at WSMR against LightSquared/Ligado base station interference}{}{}

\cventry{2010-2015}{Software Engineer}{COSMIC-II}{Software architect for TriG, designed and delivered GNSS radio occultation instrument software for the COSMIC-II mission}{}{}

\cventry{2009-2011}{Firmware Engineer}{LISA}{Designed hardware/software for the LISA gravitational wave detector to perform precise phase locking of lasers to a cavity using the Pound-Drever-Hall (PDH) technique}{}{}

\cventry{2006-2008}{Software/Firmware Engineer}{TOGA}{Designed hardware, firmware, and software for a 5 FPGA beamforming GPS reflections instrument. Performed airplane test flights over Pacific ocean with instrument}{}{}


\section{Other}
\cventry{2010--present}{taxview.org}{}{In order to help people visualize better where their taxes are going, I put together a website over a few weekends: \link[]{taxview.org}}{}{}
\cventry{2004}{NRAO}{Green Bank Radio Telescope}{Worked for the NRAO at the Green Bank Telescope facility in Green Bank, WV. Wrote Python code for the 100m radio telescope, allowing the telescope to track satellites given Keplerian elements. It is the plan of the GBT facility to use this code as a warning system to alert the observer when a satellite moves into the beam of the telescope}{}{}

\section{Selected Publications}
\cvline{}{Moon-to-Earth: Eavesdropping on the GRAIL Inter-Spacecraft Time-Transfer Link using a Large Antenna and a Software Receiver: S. Esterhuizen, NASA/JPL-Caltech; Proceedings of the ION's GNSS 2013 conference.}

\cvline{}{An innovative direct measurement of the GRAIL absolute timing of Science Data: Kamal Oudrhiri, Sami Asmar, Stephan Esterhuizen, Charles Goodhart, Nate Harvey, Daniel Kahan, Gerhard Kruizinga, Meegyeong Paik, Dong Shin, Leslie White, NASA/JPL-Caltech; Proceedings of the IEEE Aerospace 2014 conference.}

\cvline{}{Experimental Characterization of Land-Reflected GPS Signals: S. Esterhuizen, D. Masters, D. Akos, E. Vinande, University of Colorado. Proceedings of the ION's GNSS 2005 conference.}

\cvline{}{Analysis of GNSS Signals as Observed Via a High Gain Parabolic Antenna: M. Pini, Politecnico di Torino, Italy; S. Esterhuizen, D. Akos, University of Colorado. Proceedings of the ION's GNSS 2005 conference.}

\cvline{}{Integration of GNSS Bistatic Radar Ranging into an Aircraft Terrain Awareness and Warning System: D. Masters, P. Axelrad, D. Akos, S. Esterhuizen, University of Colorado. Proceedings of the ION's GNSS 2005 conference.}

\cvline{}{The Use of GPS Bistatic Radar Techniques with a Software GPS Receiver for Terrain Elevation Measurements and Ground Object Detection: E. Vinande, D. Akos, D. Masters, P. Axelrad, S. Esterhuizen, University of Colorado, Boulder. Proceedings of the ION's 61st Annual Meeting, 2005.}

\cvline{}{High Gain Antenna Measurements and Signal Characterization of the GPS Satellites: D.M. Akos, S. Esterhuizen, University of Colorado; A. Mitelman, R.E. Phelts, P. Enge, Stanford University. Proceedings of the ION's GNSS 2004 conference.}

\section{Patents}
\cvline{}{A Method to measure total noise temperature of a wireless receiver during operation. United states patent 8688065, Issued April 2014. Lawrence E. Young, Dmitry Turbiner, Stephan X. Esterhuizen.}

\section{Group Awards}
\cvline{2014}{TriG Flight Science Instrument Development Team}
\cvline{2013}{GRAIL Timing Synchronization Team}
\cvline{2013}{MSL Telecommunications Development Team}
\cvline{2013}{DSN-Parkes MSL EDL Support Team}
\cvline{2012}{GRAIL Science Data System Team}
\cvline{2012}{GRAIL TTA DTE Team}
\cvline{2011}{GPS Interference Testing Team}
\cvline{2011}{CoNNeCT Post-Delivery GPS Team}

\end{document}

