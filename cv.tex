\documentclass[11pt,letter]{moderncv}

\moderncvtheme[blue]{classic} 
%\usepackage[utf8]{inputenc}  %Windows 

%\usepackage[scale=0.975]{geometry}
\usepackage[top=1.3cm, bottom=1.3cm, left=0.8cm, right=0.8cm]{geometry}
\usepackage{graphicx}

\firstname{Stephan X.}
\familyname{Esterhuizen}
\title{GNSS-R Payload Manager}         
\address{Spire Global}{}   
\mobile{+1-818-949-8407}
%\phone{}                
%\fax{Votre Fax}                      
%\email{stephan.esterhuizen@jpl.nasa.gov}
\extrainfo{}
%\photo[50pt]{photo.png} 
%\quote{Objective: To work on interesting problems for missions around-and-beyond Earth.}
\makeatletter
\renewcommand*{\bibliographyitemlabel}{\@biblabel{\arabic{enumiv}}}
\makeatother

%\usepackage{multibib}
%\newcites{book,misc}{{Books},{Others}}

\nopagenumbers{}                         
\begin{document}
\maketitle
\section{Education}
\cventry{2004--2006}{M.S. Electrical Engineering}{University of Colorado}{Boulder}{}{}
\cventry{2000--2004}{B.S. Computer Engineering}{University of Colorado}{Boulder}{}{}

\section{Skills}
\cvline{Languages}{C/C++, Python, Matlab, Bash, Verilog, Assembly, JavaScript, HTML/CSS, SQL}
\cvline{Other}{FPGA, Jenkins, Vivado, HLS, Linux, Yocto, uBoot, VxWorks, RTEMS, Github, OpenGL, Blender, ZeroMQ}

\section{Spire Global (2018-present)}
\cventry{2018--present}{Payload Manager}{GNSS-R}{Working with a team of engineers to deliver a GNSS-R instrument for small satellites}{}{}

\section{NASA/JPL (2006-2018)}
\cventry{2017--2018}{Co-Investigator}{GNSS-R}{NASA IIP: Part of a team building a multi-frequency beam-steerable GNSS-R instrument for small spacecraft. Receiver architect and DSP/software lead}{}{}
\cventry{2015--2017}{Principal Investigator}{8x R\&TD}{Towards sub-millimeter formation knowledge and millimeter-level formation control for small satellites. Developing software, hardware, and algorithms to enable fuel-efficient formation flight with small spacecraft.}{}{}
\cventry{2015--2016}{Co-Investigator}{GNSS-R SMAP}{Received Spontaneous R\&D money, developed signal processing to detect Earth-reflected GPS L2 signals using SMAP radar receiver (post SMAP transmitter anomaly)}{}{}
\cventry{2015--2016}{Technical Team Lead}{GNSS-RO CICERO}{Led a team of 5 engineers to deliver a radio occultation instrument for the CICERO cube-sat mission. On-time and on-budget delivery of instrument}{}{}
\cventry{2014--2017}{Co-Investigator}{GNSS-R/RO}{Designed a new wide-band Radio Frequency (RF) ASIC front end for GNSS remote sensing  applications. Architect of optimal RF ASIC for radiometric performance}{}{}
\cventry{2012--2015}{US. Delegate}{}{Part of the US delegation representing NASA at the United Nations Office of Outer Space Affairs, International Committee on GNSS. Advising UN member states on designing inter-operable GNSS constellations with the scientific community in mind}{}{}
\cventry{2010-2015}{Software Engineer}{COSMIC-II}{Software architect for TriG, designed and delivered GNSS radio occultation instrument software for the COSMIC-II mission}{}{}
\cventry{2009-2011}{Firmware Engineer}{LISA}{Designed hardware/software for the LISA gravitational wave detector to perform precise phase locking of lasers to a cavity using the Pound-Drever-Hall (PDH) technique}{}{}
\cventry{2006-2008}{Software/Firmware Engineer}{GNSS-R}{Designed hardware, firmware, and software for TOGA, a 5 FPGA beamforming GPS reflections instrument. Performed airplane test flights over Pacific ocean with instrument}{}{}

\section{Patents}
\cvline{}{A Method to measure total noise temperature of a wireless receiver during operation. United states patent 8688065, Issued April 2014. Lawrence E. Young, Dmitry Turbiner, Stephan X. Esterhuizen.}

\end{document}

