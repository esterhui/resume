\documentstyle[margin,line]{res}

% Logos
% =====

\font\attand=cmr7
%\font\MetafontLogoFont=logo10

\def\PS{{\tt P\small OST\tt S\small CRIPT}}
\def\ATT{{AT{\attand \&}T}}
\def\MF{{METAFONT}}
\def\Cplusplus{{\rm C\raise.5ex\hbox{\small ++}}}
\def\AmSTeX{{$\cal A\kern-.1667em\lower.5ex\hbox{$\cal M$}\kern-.125em
S$-\TeX}}

\oddsidemargin -.5in
\evensidemargin -.5in
\textheight 10in
\textwidth=6.0in

\begin{document}


\name{Stephan Esterhuizen\footnote{US Citizen}}
\address{202 S. Raymond Ave Unit 511\\
Pasadena, CA 91105 - Stephan.Esterhuizen@Colorado.EDU - (818)949-8407}

\begin{resume}

\begin{format}
\title{l}\employer{r}\\
\location{l}\dates{r}\\
\body\\
\end{format}

\section{\sc Objective}
I am currently very happy at NASA/JPL, the objective here is to explore the
arena and see if there are other engineering projects available that might
spark my interest. 
\section{\sc Education}
{\bf Electrical Engineering, M.S.} May, 2006\\
University of Colorado, Boulder, CO. \newline
Cumulative GPA: {\bf 3.67} \\

{\bf Electrical and Computer Engineering, B.S.} May, 2004\\
University of Colorado, Boulder, CO. \newline
Cumulative GPA: {\bf 3.52}\\


%%%%%%%%%%%%%%%%%%%%%%%%%%%%%%%%%%%%%%%%%%%%%%%%%%%%%
% I put the experience stuff here
%%%%%%%%%%%%%%%%%%%%%%%%%%%%%%%%%%%%%%%%%%%%%%%%%%%%%

\section{\sc Experience}

\title{\bf California Institute of Technology}
\employer{\bf NASA/Caltech/JPL}
\dates{\bf 2006 -- present}
\location{Jet Propulsion Lab}
\begin{position}
I was hired at JPL as a member of the technical staff, interacting with a small group of about 15 engineers building scientific instruments for spacecraft. It is the task of our group to build precise positioning instruments using GPS (or custom solutions where GPS is not available, eg. when orbiting Earth's moon). My position in the group varies depending on what stage we are in the design of the instrument. Currently I'm working on RF/Microwave (1GHz-2GHz) electronics, once the circuit board is complete, I will transition to writing DSP firmware/software to control the instrument for the onboard FPGA and PowerPC chips.
\end{position}

\title{{\bf taxview.org}}
\employer{\bf self}
\dates{\bf 2010 -- present}
\location{Visualize your taxes}
\begin{position}
In order to help people visualize better where their taxes are going, I threw together a website over a few weekends: {\bf taxview.org}. This allowed me to pick up javascript and learn the python django web frameworks.
\end{position}

\title{\bf Colorado Center for Astrodynamics Research}
\employer{\bf CCAR}
\dates{\bf 2003 -- 2006}
\location{UCB}
\begin{position}
Research Assistant at CCAR, work focus on the analysis of reflected GPS signals for bistatic radar applications (mainly altimetry). This work has opened up the opportunity to explore the many facets of GPS, including {\bf DSP}, {\bf CDMA}, {\bf ADCs}, {\bf software receivers},{\bf  analog front ends}, {\bf remote sensing}, {\bf high-speed USB data bridges}, and {\bf weak signal tracking}. 

Earlier work at CCAR included writing of OpenGPS, an {\bf open source GPS package} that runs on {\bf real-time Linux}. The code currently reflects a very modular design and allows for real-time visualization of received GPS signals. This software package is currently used by the University to help students understand the inner workings of a GPS receiver. This software was also used to assist in the teaching of a Navtech course in Long Beach during the 2005 Institute of Navigation (ION) conference. 
\end{position}

\title{\bf AESNet}
\employer{\bf AES}
\dates{\bf 2005 -- 2006}
\location{UCB}
\begin{position}
Designing and building a {\bf 10-computer Linux cluster} for the Aerospace Department at the University (called AESNet). The AESNet project's goal is to build a {\bf scalable computer network} that minimizes administration costs and support of the network while providing a superior computing environment for its users. This computing environment will feature clustering capabilities as well as provide access to a wide range of scientific/engineering software packages. \\\\
\end{position}

\title{\bf COSMIC}
\employer{\bf UCAR}
\dates{\bf 2005 (3 months)}
\location{UCB}
\begin{position}
Worked part-time for University Corporation for Atmospheric Research (UCAR) during 2005 building a {\bf GPS bitgrabber system} (based on OpenGPS). The system will support the Formosat/COSMIC mission; UCAR plans to deploy 8 of these bitgrabber systems world wide in order to collect the 50Hz data message that the GPS satellites transmit. This data stream is sent back to UCAR for real-time processing. 
\end{position}

\title{\bf National Radio Astronomy Observatory}
\employer{\bf NRAO}
\dates{\bf 2004 (1 Month)}
\location{Green Bank, WV}
\begin{position}
Worked for the NRAO at the Green Bank Telescope facility in Green Bank, WV. Wrote {\bf python code for the 100m radio telescope}, allowing the telescope to track satellites given Keplerian elements. It is the plan of the GBT facility to use this code as a warning system to alert the observer when a satellite moves into the beam of the telescope. 
\end{position}

\title{\bf Senior Project - UAV Avionics}
\employer{\bf ECE Department}
\dates{\bf 2002 -- 2004}
\location{UCB}
\begin{position}
Member of Electrical Team for Uncrewed Aerial Vehicle (UAV). Team designed a
distributed system based on the CAN bus and {\bf 8-bit Atmel AVR RISC microcontroller}. This PCB design has since been used to control UAVs and as data collection systems for Aerospace senior projects. 
\end{position}

\title{\bf Colorado Space Grant Consortium}
\employer{\bf CSGC}
\dates{\bf 2000 -- 2003}
\location{UCB}
\begin{position}
Team lead for Three Corner Satellite Hardware Team and Citizen Explorer C\&DH
team. Designed embedded systems for satellites, including working with {\bf 8051/PPC823 microcontrollers}. Get more info at http://spacegrant.colorado.edu.
\end{position}

%\title{\bf Engineering Excellence Fund}
%\employer{\bf EEF}
%\dates{\bf 2001 -- 2006}
%\location{UCB}
%\begin{position}
%The EEF committee is student-run and allocate \$800,000 worth of grants every academic year to Engineering projects. Member since 2001 and committee chair for AY 2003 and 2004. Get more info at http://eef.colorado.edu.
%\end{position}

%\title{\bf National Science Expo}
%\employer{\bf Menlopark High School}
%\dates{\bf 1999}
%\location{Pretoria, South Africa}
%\begin{position}
%Gold medallist in National Science Expo. Designed and built a working model for
%stabilising a suspended mass. Published work in South Africa's national science
%magazine {\em Archimedes, Volume 41, no. 1, Summer 1999} ``Stabiliser for a
%suspended mass''
%\end{position}

\section{\sc Special Skills}
{\bf Computer Skills}: PCB layout, \LaTeX, Python, C, C++, PostgreSQL, Matlab,
	RISC assembly.\\ 
{\bf Operating Systems}: Linux, RTLinux, VxWorks\\
{\bf Computer Architectures}: AVR RISC, 8051, USB/FX2, PPC823, x86\\
{\bf Languages}: English, Afrikaans

\section{\sc Honors, Awards, and other Activities}
Dean's list Fall 2000, Fall 2001, Spring 2003\\
2003 Institute of Navigation (ION) Rocky Mountain Section Scholarship\\
2001 Mohammed H. Zanboorie Memorial Scholarship\\
Gold medalist in National Science Expo, South Africa
Group leader for High School Honors Institute

\section{\sc Publications}
{\bf Experimental Characterization of Land-Reflected GPS Signals}: S. Esterhuizen, D. Masters, D. Akos, E. Vinande, University of Colorado. Proceedings of the ION's GNSS 2005 conference.

{\bf Analysis of GNSS Signals as Observed Via a High Gain Parabolic Antenna}: M. Pini, Politecnico di Torino, Italy; S. Esterhuizen, D. Akos, University of Colorado. Proceedings of the ION's GNSS 2005 conference.

{\bf Integration of GNSS Bistatic Radar Ranging into an Aircraft Terrain Awareness and Warning System}: D. Masters, P. Axelrad, D. Akos, S. Esterhuizen, University of Colorado. Proceedings of the ION's GNSS 2005 conference.

{\bf The Use of GPS Bistatic Radar Techniques with a Software GPS Receiver for Terrain Elevation Measurements and Ground Object Detection}: E. Vinande, D. Akos, D. Masters, P. Axelrad, S. Esterhuizen, University of Colorado, Boulder. Proceedings of the ION's 61st Annual Meeting, 2005.

{\bf High Gain Antenna Measurements and Signal Characterization of the GPS Satellites}: D.M. Akos, S. Esterhuizen, University of Colorado; A. Mitelman, R.E. Phelts, P. Enge, Stanford University. Proceedings of the ION's GNSS 2004 conference.
\end{resume}
\end{document}
